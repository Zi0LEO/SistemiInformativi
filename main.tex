\documentclass{article}
\usepackage[utf8]{inputenc}
\usepackage{tabularx}
\usepackage{geometry}
 \geometry{
 a4paper,
 total={170mm,257mm},
 left=20mm,
 top=20mm,
 }
 \usepackage{graphicx}
 \usepackage{titling}
 \usepackage{cmbright}
 \usepackage{imakeidx}
 \makeindex

\author{Umberto Frega, Leonardo Napoli}
\setlength{\headheight}{12.49998pt}
\setlength{\footskip}{46.66675pt}
\addtolength{\topmargin}{-0.49998pt}
 
 \usepackage{fancyhdr}
 \pagestyle{fancy}
 \fancyhf{}
 \fancyfoot[R]{\includegraphics[width=1.5cm]{assets/logo.png}}
 \fancyhead[L]{OPCal}
 \fancyhead[R]{\theauthor}

\begin{document}

\begin{titlepage}
    \centering
    {\Large \bfseries Progetto di MODULO 2: Laboratorio di Sistemi Informativi\par}
    {\Large Anno Accademico 2024/2025 \par}
    \vspace{1cm} % Adjust if necessary
    \vfill
    {\huge Sistema Informativo per l'Organizzazione Postale Calabrese\par}
    \vfill
\noindent
\begin{minipage}[t]{0.5\textwidth}
    \raggedright
     Docente \\  prof. Francesco Parisi
\end{minipage}%
\hfill
\begin{minipage}[t]{0.4\textwidth}
    \raggedleft
     Studenti \\ Umberto Frega 239527 \\ Leonardo Napoli 234364
\end{minipage}
    
\end{titlepage}

\section{Introduzione}
L'Organizzazione Postale Calabrese (OPCal) affiliato a Poste Italiane, con sede legale a Cosenza e filiale a Rende(CS), consiste in un Ufficio atto alla spedizione e ricezione di corrispondenze nonchè all'erogazione di servizi bancari, quali pagamenti di bollette o bonifici. 
\subsection{Sede}
La sede di Rende(CS) in via Marconi, 11 è una piccola struttura avente 3 sportelli e un magazzino. La sua clientela è composta prevalentemente da studenti della vicina Università della Calabria che effettuano operazioni di ricevimento pacchi. La fondazione dell'ufficio risale al 2017, pertanto l'attuale sistema informativo è datato. 
\subsection{Organizzativo}
Il direttore dal 2020 è il sig.Lucio Dalla. L'ufficio è diviso in 3 sezioni, la sezione Sportello con 3 dipendenti è gestita dal sig.Francesco de Gregori, la sezione Magazzino con 2 dipendenti è gestita dal sig.Adriano Celentano e la sezione Recapiti è gestita dal sig.Francesco Guccini con 2 dipendenti al suo seguito. In totale l'organizzazione ammonta a 11 dipendenti.
\begin{figure}[h]
    \centering
    \includegraphics[width=0.8\linewidth]{assets/organigramma.jpg}
    \caption{Organigramma}
    \label{fig:enter_label}
\end{figure}
\subsection{Posizione nel Mercato}
L'OPCal detiene ad oggi gran parte del palcoscenico postale cosentino, i competitor sono per la maggioranza servizi privati in rapida ascesa.
\subsection{Prospettive Future}
Nel breve termine l'obiettivo della OPCal rimane il mantenere le quote di mercato nell'area metropolitana cosentina con uno sguardo verso l'esterno, con la possibilità a medio/lungo termine di espandere il proprio mercato all'intera area calabrese, continuando a garantire una politica di serietà e velocità nel servizio e disponibilità del personale. 
\subsection{Benefici Attesi}
L'implementazione del sistema all'interno dell'organizzazione aziendale porterà diversi benefici, come: 
\begin{itemize}
    \item Rinnovamento del sistema attuale;
    \item Centralizzazione delle funzionalità;
    \item Centralizzazione dei dati;
    \item Rimozione della dipendenza da documenti cartacei;
    \item Maggiore possibilità di estensione dell'organizzazione;
    \item Semplificazione e deburocratizzazione della user experience;
    \item Possibilità di avere un sistema pubblicitario più esteso ed efficiente tramite il sito web; 
    
\end{itemize}
\subsection{Funzionalità}
Le macro-funzionalità fornite dal sistema infomativo sono le seguenti:
\begin{enumerate}
    \item \textbf{Gestione dei Clienti} \begin{itemize}
        \item Il sistema avrà la funzionalità di mantenere, organizzare e visualizzare le informazioni riguardanti i clienti, quali \textit{spedizioni a carico}, \textit{saldo corrente} e \textit{pacchi in arrivo}.
        \item Ogni cliente avrà la possibilità di visualizzare i suoi dati, prenotarsi allo sportello e prenotare una spedizione tramite l'apposito sito web oppure l'applicativo per dispositivi mobile.  
        \item L'utente potrà gestire il proprio saldo corrente e decidere come utilizzarlo.
    \end{itemize}
    \item \textbf{Gestione dei Dipendenti} \begin{itemize}
        \item Il sistema terrà traccia dei vari dipendenti e delle operazioni che svolgono durante il giorno, nonchè dei loro turni e delle loro buste paga.
        \item Il responsabile di ciascun settore potrà visualizzare le prestazioni dei propri subordinati, allo scopo di ammonire e/o premiare i dipendenti meritevoli tramite un sistema di punti.
        \item Ogni dipendente potrà accedere alle informazioni riguardanti la propria busta paga e i turni che dovrà rispettare.
    \end{itemize}
    \item \textbf{Gestione dei Recapiti} \begin{itemize}
        \item Il sistema terrà traccia dello stato delle spedizioni prenotate, in corso ed effettuate in una base di dati.
        \item La base di dati interagirà con l'interfaccia fornita ai dipendenti che si occuperanno di aggiornare lo stato delle spedizioni.
        \item Per ogni missiva presa in carico il sistema sarà responsabile di associarle un codice identificativo univoco a 6 cifre che contraddistinguerà l'oggetto dall'inizio alla fine della sua lavorazione,  
    \end{itemize}
    \item \textbf{Gestione del Magazzino} \begin{itemize}
        \item Il magazzino interagirà con il sistema tramite una base di dati.
        \item La gestione del magazzino sarà strettamente legata alla gestione dei recapiti, in quanto il magazzino contiene le corrispondenze in entrata e in uscita.
        \item Sarà possibile gestire ogni missiva tramite il codice e trovarla facilmente nonchè catalogarla in base ai suoi dati.
        \item Per ogni oggetto nel magazzino sarà anche memorizzata la sua posizione negli scaffali.
    \end{itemize}
    \item \textbf{Gestione Pagamenti} \begin{itemize}
        \item Il sistema avrà la funzione di monitoraggio dei pagamenti sia verso l'azienda OPCal che nei confronti di enti esterni. 
        \item Il cliente potrà visualizzare lo stato dei propri pagamenti nonchè lo stato delle sue scadenze in dirittura.
        \item Il sistema interagirà con i vari emittenti di bollette, imposte, bollettini per garantire un servizio continuativo e centralizzato di pagamento. 
    \end{itemize}
    \item \textbf{Interfaccia} \begin{itemize}
    \item Il sistema provvede un'interfaccia per utenti e dipendenti.
    \item Tale interfaccia avrà una duplice implementazione, un sito web e un applicativo per dispostivi mobili (Android o IOS). 
    \item Le funzionalità esposte al pubblico saranno tutte disponibili tramite le interfacce specificate,
    \end{itemize}
\end{enumerate}

\newpage
\section{Analisi dei Requisiti}
\subsection{Analisi dello scenario}
 
\begin{tabularx}{\textwidth}{
    |>{\centering\arraybackslash}X
    |>{\centering\arraybackslash}X
    |>{\centering\arraybackslash}X
    |>{\centering\arraybackslash}X
    |>{\centering\arraybackslash}X|
  }
  \hline
  \textbf{Logistica in entrata(LE)} & \textbf{Attività operative(AO)} & \textbf{Logistica in uscita(LU)} & \textbf{Marketing e vendita(MV)} & \textbf{Servizi post-vendita} \\ 
  \hline
  Row 1, Col 1 & Row 1, Col 2 & Row 1, Col 3 & Row 1, Col 4 & Row 1, Col 5 \\
  \hline
\end{tabularx}








\printindex
\end{document}
